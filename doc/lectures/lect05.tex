\section{Лекция 5 (09.03)}
\subsection{Решение СЛАУ}

В рассмотренных ранее примерах использовался алгебраический
многосеточный итерационный решатель, который имеет существенное время
инициализации. Ниже рассмотрим некоторые более простые итерационные способы решения систем
уравнений, которые, хотя и имеют значительно худшую сходимость, но не требуют дорогой инициализации.

\subsubsection{Метод Якоби}
\label{sec:SLAE-Jacobi}

Будем рассматривать систему уравнений вида
\begin{equation*}
    \sum_{j=0}^{N-1} A_{ij} u_j = r_i, \quad i = \overline{0, N-1}
\end{equation*}
относительно неизвестного сеточного вектора $\{u\}$.

В классическом виде алгоритм Якоби формулируется в виде
\begin{equation*}
    \hat u_i = \frac{1}{A_{ii}}\left(r_i - \sum_{j\neq i} A_{ij}{u_j}\right)
\end{equation*}

Произведём некоторые преобразования
\begin{align*}
    \hat u_i &= \frac{1}{A_{ii}}\left(r_i - \sum_{j} A_{ij}{u_j} + A_{ii}u_i\right) \\
             &= u_i + \frac{1}{A_{ii}}\left(r_i - \sum_{j} A_{ij}{u_j}\right)
\end{align*}

Таким образом, программировать итерацию этого алгоритма, обновляющую значения массива $\{u\}$, можно в виде
\begin{align*}
    &\hat u = u; \\
    &\textbf{for } i=\overline{0, N-1} \\ 
    &\quad \hat u_i \mathrel{{+}{=}} \frac{1}{A_{ii}}\left(r_i - \sum_{j=0}^{N-1} A_{ij}{u_j}\right)\\
    &\textbf{endfor} \\
    & u = \hat u; \\
\end{align*}


\subsubsection{Метод Зейделя}
\label{sec:SLAE-Seidel}
Формулируется в виде
\begin{equation*}
    \hat u_i = \frac{1}{A_{ii}}\left(r_i - \sum_{j<i} A_{ij}{\hat u_j} - \sum_{j>i} A_{ij}{u_j} \right).
\end{equation*}

Поскольку этот метод неявный относительно уже найденных на итерации значений, то в отличии от метода Якоби этот алгоритм не требует создания временного массива $\hat u$
при программировании. Псевдокод для реализации итерации этого метода можно записать как

\begin{align*}
    &\textbf{for } i=\overline{0, N-1} \\ 
    &\quad u_i \mathrel{{+}{=}} \frac{1}{A_{ii}}\left(r_i - \sum_{j=0}^{N-1} A_{ij}{u_j}\right)\\
    &\textbf{endfor}
\end{align*}


\subsubsection{Метод последовательных верхних релаксаций (SOR)}
\label{sec:SLAE-SOR}
Этот метод основан на добавлении к решению результатов итераций Зейделя с
коэффициентом $\omega > 1$. То есть он изменияет решение по тому же
принципу, что и метод Зейделя, но искуственно увеличивает эту добавку.

Формулируется этот метод в виде
\begin{equation*}
    \hat u_i = (1-\omega) u_i + \frac{\omega}{A_{ii}}\left(r_i - \sum_{j<i} A_{ij}{\hat u_j} - \sum_{j>i} A_{ij}{u_j} \right).
\end{equation*}
Для устойчивости метода необходимо $\omega < 2$. В частности, для одномерных задач,
заданных на единичном отрезке, для оптимальной сходимости можно использовать соотношение $\omega \approx 2 - 5 h$,
где $h$ -- шаг сетки.

Итерация этого метода по аналогии с методом Зейделя может быть запрограммирована в виде

\begin{align*}
    &\textbf{for } i=\overline{0, N-1} \\ 
    &\quad u_i \mathrel{{+}{=}} \frac{\omega}{A_{ii}}\left(r_i - \sum_{j=0}^{N-1} A_{ij}{u_j}\right)\\
    &\textbf{endfor}
\end{align*}

\subsection{Задание для самостоятельной работы}

Вернутся к рассмотрению двумерного уравнения Пуассона (п.~\ref{sec:hw_poisson_2d}).
Прошлая реализация этой задачи
включала в себя решение СЛАУ алгебраическим многосеточным методом
с помощью класса \cvar{AmgcMatrixSolver}:
\begin{cppcode}
	AmgcMatrixSolver solver;
	solver.set_matrix(mat);
	solver.solve(rhs, u);
\end{cppcode}

Необходимо реализовать рассмотренные ранее методы
итерационного решения СЛАУ

\begin{itemize}
\item метод Якоби (\ref{sec:SLAE-Jacobi}),
\item метод Зейделя (\ref{sec:SLAE-Seidel}),
\item метод SOR (\ref{sec:SLAE-SOR}).
\end{itemize}
и использовать их вместо многосеточного решателя.

Реализовать означенные решатели нужно в виде функций вида:
\begin{cppcode}
// Single Jacobi iteration for mat*u = rhs SLAE. Writes result into u
void jacobi_step(const cfd::CsrMatrix& mat, const std::vector<double>& rhs, std::vector<double>& u){
    ...
}
\end{cppcode}
которые делают одну итерацию соответствующего метода без проверок на сходимость.
Аргумент \cvar{u} используется как начальное значение искомого сеточного вектора. Туда же пишется 
итоговый результат.

Все алгоритмы основаны на вычислении выражения вида
\begin{equation*}
   \frac{1}{A_{ii}}\left(r_i - \sum_{j=0}^{N-1} A_{ij}{u_j}\right),
\end{equation*}
поэтому рекомендуется выделить отдельную функцию, которая бы вычисляла это выражение
и использовалась всеми тремя решателями
\begin{minted}[linenos=false]{c++}
double row_diff(size_t irow, const cfd::CsrMatrix& mat, const std::vector<double>& rhs, const std::vector<double>& u){
	const std::vector<size_t>& addr = mat.addr();   // массив адресов
	const std::vector<size_t>& cols = mat.cols();   // массив колонок
	const std::vector<double>& vals = mat.vals();   // массив значений
    ...
}
\end{minted}

Дополнительно понадобится реализовать функцию, которая проверяет сходимость решения
путём вычисления невязки вида
\begin{equation*}
res = \max_i{\left| \sum_{j}A_{ij} u_j - r_i \right|}
\end{equation*}
и сравнения с заданным малым числом $\eps=10^{-8}$.
\begin{cppcode}
bool is_converged(const cfd::CsrMatrix& mat, const std::vector<double>& rhs, const std::vector<double>& x){
	constexpr double EPS = 1e-8;
	double residual = 0;
	// ...
	return residual < EPS;
}
\end{cppcode}

Для реализации вспомогательных функций необходимо
использоватть алгоритмы работы с CSR-матрицами из п.~\ref{sec:csr}

При реализации метода SOR подобрать оптимальный параметр $\omega$,
при котором метод SOR сойдётся за минимальное число итераций.

После реализации всех методов необходимо
сравнить время исполнения решателей. Замеры нужно проводить в Release-версии сборки.
Для замера времени исполнения участка кода воспользоваться функциями
\begin{itemize}
\item \cvar{cfd::dbg::Tic} -- вызвать до начала участка кода
\item \cvar{cfd::dbg::Toc} -- вызвать после окончания участка кода
\end{itemize}

Код решения СЛАУ методом Якоби с вызовами профилироващика
должен иметь примерно такой вид:
\begin{cppcode}
#include "dbg/tictoc.hpp"
using namespace cfd;

...

// реализация решения СЛАУ
dbg::Tic("total");  // запустить таймер total
for (size_t it=0; it < max_it; ++it){
	dbg::Tic("step");  // запустить таймер step
	jacobi_step(mat, rhs, u);
	dbg::Toc("step");  // остановить таймер step

	dbg::Tic("conv-check");  // запустить таймер conv-check
	bool is_conv = is_converged(mat, rhs, u);
	dbg::Toc("conv-check");  // остановить таймер conv-check

	if (is_conv) break;
}
dbg::Toc("total");  // остановить таймер total
\end{cppcode}

При правильном задании функций замеров, по окончанию работы в консоль должен напечататься отчёт о времени исполнения вида:
\begin{shelloutput}
     total:  6.670 sec
      step:  5.220 sec
conv-check:  1.210 sec
\end{shelloutput}

По результатам профилировки нужно заполнить таблицу

\begin{equation*}
\begin{array}{l|c|c|c|c|c|c}
    Метод & \text{total, s} & \text{step, s} & \text{conv-check, s} & \text{Кол-во итераций} \\
    \hline
    \text{Amg} &   & - & - & - \\
    \hline
    \text{Якоби} &   &  &  & \\
    \hline
    \text{Зейдель} &   &  &  & \\
    \hline
    \text{SOR} (\omega=...)&   &   & & \\
\end{array}
\end{equation*}
Здесь Amg - исходный решатель.
