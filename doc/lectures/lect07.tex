\section{Лекция 7 (30.03)}

\subsection{Граничные условия второго рода}
Учёт условий второго рода тривиален.
Если на центре грани $\Gamma_{is}$ задано 
значение нормальной производной
\begin{equation}
\label{eq:fvm_bc2_approx}
\vec x \in \Gamma_{is}: \quad \dfr{u}{n} = q(\vec x),
\end{equation}
то это значение просто подставляется вместо соответствующей производной
в  \cref{eq:fvm_gamma_integral}.

По аналогии с \eqref{eq:fvm_assem_bc1}
учёт граничных условий второго рода
при сборке по граням будет иметь следующий вид
\begin{equation}
\label{eq:fvm_assem_bc2}
\begin{array}{ll}
\textbf{for } s \in\textrm{bnd2}                         & \textrm{-- грани с условиями второго рода}\\ 
\qquad i = \textrm{nei\_cells(s)}                        & \textrm{-- соседняя с граничной гранью ячейка}\\
\qquad b_{i} \pluseq |\Gamma_{is}| q                     & \\
\textbf{endfor}                                          & \\
\end{array}
\end{equation}


\subsection{Граничные условия третьего рода}
Теперь рассмотрим условия третьего рода
\begin{equation}
\label{eq:fvm_bc3}
\vec x \in \Gamma_{is}: \quad \dfr{u}{n} = \alpha(\vec x) u + \beta(\vec x).
\end{equation}
Распишем производную в форме \cref{eq:fvm_bc1_approx}:
\begin{equation*}
\frac{u^\Gamma - u_i}{h_{is}} = \alpha u^\Gamma + \beta,
\end{equation*}
откуда выразим $u^\Gamma$:
$$
u^\Gamma =  \frac{u_i + \beta h_{is}}{1 - \alpha h_{is}}.
$$
Подставляя это выражение в исходное граничное условие \eqref{eq:fvm_bc3} получим
\begin{equation}
\label{eq:fvm_bc3_approx}
\dfr{u}{n} = \frac{\alpha}{1 - \alpha h_{is}} u_i + \frac{\beta}{1 - \alpha h_{is}}.
\end{equation}

Учёт граничных условий третьего рода при сборке по граням будет иметь вид

\begin{equation}
\label{eq:fvm_assem_bc3}
\begin{array}{ll}
\textbf{for } s \in\textrm{bnd3}                         & \textrm{-- грани с условиями третьего рода}\\ 
\qquad i = \textrm{nei\_cells(s)}                        & \textrm{-- соседняя с граничной гранью ячейка}\\
\qquad v = \sfrac{|\Gamma_{is}|}{(1 + \alpha h_{is})}    & \\
\qquad A_{ii} \minuseq  \alpha v                         & \\ 
\qquad b_{i} \pluseq \beta v                             & \\
\textbf{endfor}                                          &
\end{array}
\end{equation}

\subsubsection{Универсальность условий третьего рода}
Условие третьего рода
\eqref{eq:fvm_bc3}
можно использовать для моделирования условий первого и второго рода.
Так, условия второго рода \cref{eq:fvm_bc2_approx} получаются, если положить $\alpha = 0$, $\beta = q$.
А условия первого \cref{eq:fvm_bc1}, -- если
\begin{equation}
\label{eq:fvm_bc3_universal}
\alpha = \eps^{-1}, \quad \beta = -\eps^{-1}u^\Gamma
\end{equation}
, где $\eps$ -- малое положительное число.

Если подставить эти выражения в формулу \cref{eq:fvm_bc3_approx}, то можно убедится,
что они дадут выражения \cref{eq:fvm_bc1_approx} и \cref{eq:fvm_bc2_approx} (в пределе при $\eps \to 0$)
соответственно.

\subsection{Задание для самостоятельной работы}
Сделать задачу, аналогичную п.~\ref{sec:hw_fvm2d}, 
но использовать граничные условия 3-его рода.
Необходимо имитировать условия первого рода через подход (\ref{eq:fvm_bc3_universal}).

По формуле
$$
n = \sqrt{\dfrac{\sum_i (u_i - u'_i)^2 V_i}{\sum_i V_i}}
$$
подсчитать норму отклонения численного решения $u_i$ задачи с использованием истинных граничных условий первого рода \eqref{eq:fvm_bc1_approx}
от численного решения $u'$ задачи, расчитанной с имитицией граничных условий первого рода через граничные условия третьего рода
\cref{eq:fvm_bc3_approx,eq:fvm_bc3_universal}.

Нарисовать график $n(\eps)$ для расчётов на структурированной и скошенной сетках (в логарифмических координатах).
