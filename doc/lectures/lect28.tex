\section{Лекция 28 (19.04)}

\subsection{Стабилизация методом наименьших квадратов}

Для уравнения общего вида
\begin{equation*}
A x = b
\end{equation*}
слабая постановка со стабилизацией методом наименьших квадратов 
имеет вид
\begin{equation*}
\feint{(Ax - b)\phi} + \sum_k \gamma_k\arint{(Ax-b)A\phi}{E_k}{\vec x} = 0,
\end{equation*}
где первое слагаемое - следствие аппроксимации Галёркина, а второе -- стабилизирующее.

Рассмотрим нестационарное уравнение конвекции диффузии:
\begin{equation*}
\dfr{u}{t} + \vec v\cdot \nabla u - \eps \nabla^2 u = 0
\end{equation*}
Для него слабая постановка примет вид
\begin{equation*}
\feint{
\left(
\dfr{u}{t} + \vec v\cdot \nabla u - \eps \nabla^2 u
\right)\phi} +
\sum_k \gamma_k\arint{\left(
\dfr{u}{t} + \vec v\cdot \nabla u - \eps \nabla^2 u
\right)\left(
\dfr{\phi}{t} + \vec v\cdot \nabla \phi - \eps \nabla^2 \phi
\right)}{E_k}{\vec x} = 0.
\end{equation*}
Поскольку пробная функция $\phi$ не зависит от времени, то постановка упростится до
\begin{equation*}
\feint{
\left(
\dfr{u}{t} + \vec v\cdot \nabla u - \eps \nabla^2 u
\right)\phi} +
\sum_k \gamma_k\arint{\left(
\dfr{u}{t} + \vec v\cdot \nabla u - \eps \nabla^2 u
\right)\left(
\vec v\cdot \nabla \phi - \eps \nabla^2 \phi
\right)}{E_k}{\vec x} = 0.
\end{equation*}
Более того, для линейных конечных элементов внутри элемента справедливо $\nabla^2 \phi = 0$ и
стабилизация методом наименьших квадратов сводится к стабилизации методом SUPG.

Далее используем двухслойную схему по времени со степенью неявности $\theta$.
Нестационарное слагаемое примет вид
\begin{equation*}
\dfr{u}{t} = \frac{\hat u - u}{\tau}, 
\end{equation*}
а все вхождения $u$ распишутся в виде
\begin{equation*}
u = \theta \hat u + (1 - \theta ) u.
\end{equation*}

Окончательно систему линейных уравнений на временном слое получим
заменив $\phi$ на $\phi_i$ и
расписав $u$ (и также $\hat u$) по базису $\phi$:
\begin{equation*}
u = \sum_j u_j \phi_j.
\end{equation*}

Параметр $\gamma_k$ будем определять из следующей таблицы

\begin{equation*}
\begin{array}{l|l|l}
             & P_1  &  P_n \\
\hline
\theta = 0   & \left(                 \frac{2 |\vec v|}{h}                     \right)^{-1}     &  \left(                 \frac{2 |\vec v|}{h} + \frac{4 \eps}{h^2}\right)^{-1}   \\[10pt]
\hline
\theta = 1   & \left(\frac{1}{\tau} + \frac{2 |\vec v|}{h}                     \right)^{-1}     & \left(\frac{1}{\tau} + \frac{2 |\vec v|}{h} + \frac{4 \eps}{h^2}\right)^{-1}    \\[10pt]
\hline
\theta = 0.5 & \left(\frac{2}{\tau} + \frac{2 |\vec v|}{h}                     \right)^{-1}     & \left(\frac{2}{\tau} + \frac{2 |\vec v|}{h} + \frac{4 \eps}{h^2}\right)^{-1}
\end{array}
\end{equation*}
здесь $P_{1/n}$ обозначают элементы первой/высокой степени.

\subsection{Задание для самостоятельной работы}
В тестовом примере \cvar{[convdiff-fem-gsl]} из файла \ename{convdiff_fem_test.cpp}
производится численного решение одномерного нестационарного уравнения конвекции-диффузии
$$
\dfr{u}{t} + v \dfr{u}{x} - \eps \dfrq{u}{x} = 0
$$
в области $x\in[0, 4]$
с точным решением вида
$$
u^e(x, t) = \frac{1}{\sqrt{4\pi \eps (t + t_0)}} \exp\left(-\frac{(x - v t)^2}{4\eps(t+t_0)}\right)
$$
Точное решение используется для формулировки начальных ($t=0$) и граничных ($x=0,4$) условий первого рода.
Результат расчёта сохраняется в файл \ename{convdiff-gsl.vtk.series}.
Тестовый пример в целом аналогичен рассмотреному ранее в п.~\ref{sec:hw_supg}.
Численная схема зависит от трёх параметров:
степени базисных функций \cvar{npower},
степени неявности $\theta$
и дополнительного множителя перед $\gamma_k$  \cvar{gamma_mult}.
По окончании расчёта на момент времени $t=2$ программа печатает полученную норму отклонения численного решения от точного.


\begin{enumerate}
\item Проиллюстрировать решение, полученное с помощью полунеявной ($\theta=0.5$) и неявной ($\theta=1$) схем с уменьшением сетки,
\item Получить зависимость $n$ от множителя \cvar{gamma_mult} для неявной и полунеявной схем для двух разных шагов сетки
\item Сравнить поведение норм решения при использовании линейных, квадтратичных и кубических элементов (\cvar{npower=1,2,3}) при 
      сгущении сетки для неявной и полунеявных схем
\item Для реализации схем высокого порядка точности необходимо уметь считать оператор Лапласа в физическом пространстве от базисной функции $\nabla^2 \phi$.
      Соответствующий множители \cvar{laplace_i, laplace_j} появляются в коде при вычислении подинтегральных выражения для стабилизирующего интеграла в \cvar{local_gsl}.
      Несмотря на то, что набор базисных функций в элементе нам известен, вычисление оператора Лапласа от этих функций 
      представляет некоторую сложность, так как функции заданы в параметрическом, а дифференцирование необходимо проводить в физическом пространстве.
      Этот оператор считается в файле \cvar{fem_element.cpp} в процедуре \cvar{IElementIntegrals::OperandArg::laplace(i)} где $i$ -- индекс базисной функции, для
      которой считается оператор Лапласа.

      Для вычисления градиента в физической плоскости мы записывали формулу \cref{eq:vec_grad_dx}. Эта формула реализована в процедуре \cvar{IElementIntegrals::OperandArg::grad_phi}.
      Сейчас оператор лапласа считается с помощью дифференцирования градиента методом конечной разности.
      Необходимо вывести точную формулу для определения оператора Лапласа в преобразованных координатах $\nabla^2_{\vec x\vec x} f$.
      Для этого нужно дважды применить формулу дифференцирования \cref{eq:vec_grad_dx}.
      Далее эту формулу необходимо упростить для случая линейного геометрического преобразования сегментов, где матрица Якоби ($1\times1$) постоянна.
      То есть получить зависимость
      \begin{equation*}
          \dfrq{f}{x} = F\left(J_{11}, \dfrq{f}{\xi}\right).
      \end{equation*}
      Полученную зависимоть вставить в метод \cvar{laplace} вместо имеющейся конечной разности.
      Получить $J_{11}$ из этого метода можно \cvar{geom->jacobi(xi).j11}, а вторую производную функции $\phi_i$ через 
      \cvar{basis->upper_hessian(xi)[i][0]}.
\end{enumerate}
