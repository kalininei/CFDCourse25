\section{Лекция 24 (22.03)}

\subsection{Стабилизация методом характеристик}
\label{sec:char-stab}

Введём обозначение
$$
u^\theta = \theta \hat u + \left(1 - \theta\right) u.
$$
Полудискретизованное уравнение конвекции-диффузии со стабилизирущим слагаемым
примет вид
\begin{equation}
\label{eq:cg_stab_multidim}
\frac{\hat u - u}{\tau} =
-\vec v \cdot \nabla u 
+\eps \nabla^2 u^\theta
+\frac{\tau}{2} \nabla\cdot \left(\left(\vec v \cdot \nabla u\right)\vec v\right)
- (1-\theta)\eps\vec v \cdot \nabla\left(\nabla^2 u\right).
\end{equation}

\subsubsection{Конечноэлементная процедура}
Далее по стандартной процедуре взвешенных невязок
домножим уравнение \cref{eq:cg_stab_multidim} на систему пробных функций $\phi_i$
и проинтегрируем по области расчёта
с применением формулы интегрирования по частям \cref{eq:partint} к трём последним слагаемым правой части:
\begin{equation*}
\begin{split}
\feint{\frac{\hat u - u}{\tau}\phi_i} = &
-\feint{\left(\vec v \cdot \nabla u \right)\phi_i} \\
&
+\eps\left(
	-\feint{\nabla u^\theta \cdot \nabla \phi_i}
	+ \febint{\dfr{u^\theta}{n}\phi_i}
\right) \\
&
+\frac{\tau}{2}\left(
	-\feint{\left(\vec v \cdot \nabla u\right)\left(\vec v\cdot\nabla\phi_i\right)}
	+\febint{\left(\vec v \cdot \nabla u\right) v_n \phi_i}
\right)\\
&
+\eps(1-\theta)\left(
\feint{\nabla^2 u \nabla\cdot\left(\phi_i \vec v\right)}
-\febint{v_n \phi_i\nabla^2 u}
\right)
\end{split}
\end{equation*}
Согласно подходу Бубнова-Галёркина
разложим искомую функцию на систему базисных
функций, равную системе пробных функций
$$
u(\vec x) = \sum_i u_i \phi_i(\vec x)
$$
Получим матричное выражение для перехода на следующий временной слой
\begin{equation}
\begin{array}{l}
\mat L \hat u = \mat R u \\
\mat L = \mat M + \tau\eps\theta\left(\mat S - \mat{B^S}\right) \\
\mat R = \mat M - \tau \mat K - \tau(1 - \theta)\eps\left(\mat S - \mat{B^S}\right)
+ \frac{\tau^2}{2}\left(-\mat{K^s} + \mat{B^{Ks}}\right)
+ \eps\tau(1 - \theta)\left(\mat G_2 - \mat{B^{G_2}}\right)
\end{array}
\end{equation}
где введена новая матрица $\mat G_2$:
\begin{equation*}
\mat G_2 = \feint{\nabla^2 \phi_j \nabla\cdot\left(\vec v \phi_i\right)},
\end{equation*}
которая обращается в ноль при использовании линейных конечных элементов.

\subsection{Задание для самостоятельной работы}
В тестовом примере \cvar{[convdiff-fem-cg]}
из файла \ename{convdiff-fem-test.cpp}
производится численного решение
той же одномерной задачи, которая рассматривалась
ранее в п.~\ref{sec:hw_supg}.
Решение производится методом конечных
элементов со стабилизацией методом
характеристик.
Задача полудискретизуется по схеме Кранка-Николсон ($\theta=\sfrac12$) c шагом по времени, вычисленным через число Куранта $\rm C = 0.5$.

Необходимо:
\begin{enumerate}
\item С помощью анимированных графиков сравнить полученное численное решение одномерной задачи (\cvar{[convdiff-fem-cg]})
      с точным решением, а также с решением, полученнным с помощью SUPG-стабилизации;
\item Сделать расчёты одномерной задачи с различными шагами по времени $\tau$.
      Продемострировать отличия в полученных решениях на анимированных графиках.
      Нарисовать зависимость нормы отлонения численного решения от точного на финальный момент времени
$$
n_2 = ||u - u^e||_2
$$
от шага по времени $\tau$.
\item Написать аналогичный тест для двумерного случая (решение в единичном квадрате) и неконстантного поля скорости:
$ \vec v = (-y + 0.5, x - 0.5) $.
Точное решение в этом случае будет иметь вид
$$
\begin{array}{ll}
\bar x = 0.5 + (x_0 - 0.5) \cos(t) - (y_0 - 0.5) \sin(t)                          & \text{-- текущее положение пика}\\
\bar y = 0.5 + (x_0 - 0.5) \sin(t) + (y_0 - 0.5) \cos(t)                          & \\
r^2 = (x - \bar x)^2 + (y - \bar y)^2                                             & \text{-- расстояние от пика}\\[10pt]
u^e = \dfrac{1}{4 \pi \eps (t+t_0)} \exp\left(-\dfrac{r^2}{4 \eps (t+t_0)}\right) & \text{-- решение}
\end{array}
$$
Использовать следующие параметры:
	\begin{itemize}
	\item начальное положение пика $x_0 = 0.8, y_0 = 0.5$,
	\item сдвиг по времени $t_0 = 0.3$,
	\item коэффициент диффузии $\eps = 10^{-3}$,
	\item временной интервал $t \in [0, 2\pi]$.
	\end{itemize}
Анимировать решение, полученное на структурированной и скошенной сетках с количеством ячеек $N \approx 5000$.
\item Для структурированной сетки нарисовать зависимость $n_2$ на момент $t = 2\pi$ в зависимости от шага сетки $h$ 
      для фиксированного ${\rm C} = 0.5$ (в качестве характерной скорости при вычислении числа Куранта использовать единицу).
\item Нарисовать зависимость $n_2$ от шага по времени $\tau$ для фиксированной двумерной сетки $N\approx5000$.
      Использовать структурированную и скошенную сетки.
\end{enumerate}
