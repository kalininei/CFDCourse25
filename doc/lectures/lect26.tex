\section{Лекция 26 (05.04)}
\subsection{Конечнообъёмная coupled схема решения уравнений Навье-Стокса}
\subsubsection{Задание для самостоятельной работы}
В тесте \ename{[cavity-fvm-coupled]} из файла \ename{[cavity_fvm_coupled_test.cpp]}
реализована Coupled схема решения задачи о стационарном течении в каверне
с применением MC ограничителя.
У программы есть единственный параметр сходимости -- шаг по фиктивному времени $\tau$.
Необходимо (предварительно установив $\eps=10^{-6}$)
\begin{itemize}
\item  Нарисовать раздельно графики сходимости для 
       $$\max\left|\dfr{u}{t}\right|,\quad
         \max\left|\dfr{v}{t}\right|,\quad
         \max\left|\dfr{p}{t}\right|$$. 
\item  Исследовать схему на оптимальный $\tau$ с точки зрения скорости сходимости с изменением шага по пространству
\item  Провести аналогичный анализ для изменений числа Рейнольдса (до 1000).
\item  Проверить полученные оптимумы на неструктурированной pebi сетке.
\item  Модифицировать алгоритм, отнеся конвективные слагаемые в уравнениях моментов полностью на прошлый временной слой.
       Проверить, как при этом изменится скорость сходимости.
\end{itemize}

Программа написана неоптимально. Как с точки зрения производительности, так и с точки зрения организации кода. Необходимо
провести рефакторинг программы. Обратить внимание на
\begin{itemize}
\item Повторяющийся код в методах \cvar{u_equation}, \cvar{v_equation},
\item Повторяющийся алгоритм при вычислении поправки Rhie-Chow в методах \cvar{continuity_equation}, \cvar{compute_un_rhie_chow},
\item Некоторые части блочной матрицы не зависят от временного слоя и могли бы быть собраны один раз
\end{itemize}
При рефакторинге необходимо следить за неизменностью резульатов тестовых проверок.
