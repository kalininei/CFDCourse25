\section{Лекция 23 (25.01)}

\subsection{Вычисление матрицы Якоби через базисные функции}
TODO

\subsection{Применение квадратурных формул для вычисления элементных интегралов}
TODO
\subsubsection{Диагонализация матрицы масс}
TODO

\subsection{Матричный алгоритм построения конечных элементов высокого порядка}
TODO

\subsection{Неполные конечные элементы}
TODO

\subsection{Задание для самостоятельной работы}
Провести сравнительный анализ на порядок аппроксимации решения двумерного
уравнения Пуассона с граничными условиями первого рода
следующих конечноэлементных схем
\begin{enumerate}
\item
Линейные треугольные элементы (\figref{fig:triangle_basis_points}а)
\item
Квадратричные треугольные элементы (\figref{fig:triangle_basis_points}б)
\item
Кубические треугольные элементы (\figref{fig:triangle_basis_points}в)
\item
Неполные 9-узловые кубические треугольные элементы (\figref{fig:triangle_basis_points}г). В качестве базисных функций использовать полиномы,
составленные из 9 слагаемых вида
$$
P(\xi, \eta) \in
\rm{span}\left\{1, \xi, \eta, \xcancel{\color{gray} \xi \eta}, \xi^2, \eta^2, \xi^2\eta, \xi\eta^2, \xi^3, \eta^3 \right\}
$$
\end{enumerate}
(из 10 слагаемых, используемых для построения кубического базиса в треугольнике,
из соображений симметрии было убрано слагаемое $\xi\eta$).

Построить кривые сходимости для всех четырёх случаев на едином графике.
Для кубических треугольных элементов найти оптимальную квадтратурную функцию.

Задание в целом аналогично заданию~\ref{sec:fem_programming_problem}.

Принципы программирования конечноэлементного решателя описаны в п.~\ref{sec:fem_programming}.
В настоящем задании была добавлена концепция численного интегрирования конечного элемента,
которая заключена в реализации \cvar{NumericElementIntegrals} интерфейса \cvar{IElementIntegrals}.

Ниже рассмотрено создание кубического треугольного элемента, где для
вычисления элементых интегралов используются гауссовые квадратуры, точные для полиномов 4-ой степени (\ename{quadrature_triangle_gauss4}).
\clisting{open}{"test/poisson_fem_solve_test.cpp"}
\clisting{pass}{"TestPoissonCubicWorker::build_fem"}
\clisting{lines-range}{"geom", "elem"}
Для понижения точности до третьей, следует использовать функцию
\ename{quadrature_triangle_gauss3} и т.д.

Все необходимые тесты находятся в файле \ename{poisson_fem_solve_test.cpp}.
Для {\bf линейных треугольных} элементов использовать тест \ename{[poisson2-fem-tri]}.

Для {\bf квадратичных треугольных} элементов -- \ename{[poisson2-fem-quadratic]}.

Для {\bf кубических треугольных} элементов -- \ename{[poisson2-fem-cubic]}.

Для {\bf неполных кубических треугольных} элементов систему локальных базисных функций нужно
вычислить самостоятельно используя алгоритм из п.\ref{sec:triangle_bases}.
На основе полученных соотношений нужно создать класс
\begin{cppcode}
class TriangleCubicBasis_NoXiEta: public IElementBasis{
public:
	size_t size() const override;
	std::vector<Point> parametric_reference_points() const override;
	std::vector<BasisType> basis_types() const override;
	std::vector<double> value(Point xi) const override;
	std::vector<Vector> grad(Point xi) const override;
};
\end{cppcode}
и реализовать все необходимые функции:
\begin{itemize}
\item \cvar{size} -- общее количество базисных функций. Здесь будет девять,
\item \cvar{parametric_reference_points} -- параметрические координаты девяти узловых точек (соблюдая порядок локальной индексации),
\item \cvar{basis_types} -- типы базисных функций. Здесь все базисы узловые (\cvar{BasisType::Nodal}),
\item \cvar{value} -- значение девяти базисных функций в заданной параметрической точке. Здесь нужно подставить полученные при вычислении базисы,
\item \cvar{grad} -- градиенты вычисленных базисных функций. Для заполнения нужно подсчитать аналитические производные по $\xi$ и $\eta$ вычислиенных базисов.
\end{itemize}
Реализовать этот класс можно взяв в качестве основы уже реализованный класс \cvar{TriangleCubicBasis} из файла \ename{cfd24/fem/elem2d/traingle_cubic.hpp}. Реализовывать следует в том же файле.
После того, как базис будет реализован, его нужно использовать в тесте \ename{poisson2-fem-cubic}:
\begin{cppcode}
auto basis = std::make_shared<TriangleCubicBasis_NoXiEta>();
\end{cppcode}
Следует убрать последний базис из таблицы связности
\begin{cppcode}
tab_elem_basis.push_back({
	bas0, bas1, bas2,
	bas3, bas4, bas5, bas6, bas7, bas8
});
\end{cppcode}
и сократить общее количество базисных функций
\begin{cppcode}
size_t n_bases = grid.n_points() + 2*grid.n_faces();
\end{cppcode}
